\documentclass[11pt]{article}
\usepackage{amsmath}
\usepackage{amsfonts}
\usepackage{amssymb}
\usepackage{cite}
\usepackage{graphicx}
\usepackage{multicol,caption}
\usepackage[margin=0.5in]{geometry}
\usepackage[stable]{footmisc}
\usepackage{comment}
\usepackage{flushend}

\bibliographystyle{unsrt}

\newenvironment{Figure}
  {\par\medskip\noindent\minipage{\linewidth}}
  {\endminipage\par\medskip}

\setlength{\columnsep}{0.25in}

\raggedbottom

\begin{document}
\newcommand{\rucl}{RuCl\textsubscript{3} }
\newcommand{\ruclnospace}{RuCl\textsubscript{3}}
\graphicspath{C:/Users/dsbjr/Documents/DissertationProposal/images}
\pagenumbering{gobble}

\Large
\begin{center}

\Huge Dissertation Proposal: Electrolyte Gating of the Two-Dimensional Proximate Kitaev Spin Liquid $\alpha$-Ruthenium Trichloride\\

\hspace{10pt}

% Author names and affiliations
\large
Derrick S. Boone, Jr.$^1$\\
	
\hspace{10pt}

\small  
$^1$ Ph.D. Candidate, Department of Applied Physics, Stanford University\\
dsbjr@stanford.edu\\

\end{center}

\section*{Executive Summary}
Quantum Spin Liquids (QSLs) are magnetic states of matter characterized by massive ground state entanglement without long-range magnetic order, even at $T = 0$ K. These materials host exotic quasiparticle excitations (e.g., Majorana fermions) and, upon doping, may demonstrate unconventional superconductivity. Previous attempts to dope the Mott insulator and Kitaev spin liquid $\alpha$-ruthenium trichloride (a 2D van der Waals material) have either disordered the lattice, added only a small number of carriers, or both. To prevent these issues, I have investigated doping $\alpha$-ruthenium trichloride using electrolyte gating with in-situ transport, Raman spectroscopy, and x-ray diffraction. I have found that the cations in the electrolyte readily intercalate between the layers, despite their large size. When controlling for the intercalation, I have not observed any electronic phase transition within the limits of electrolyte gating - in fact, resistivity either remains the same or increases. In this dissertation, I propose to show that at large hole doping without lattice disorder, $\alpha$-ruthenium trichloride does not undergo an electronic phase transition and only becomes more resistive, contrary to typical Mott insulators and the predictions for quantum spin liquids. This result suggests that the electronic structure of $\alpha$-ruthenium trichloride requires more investigation and that the material is not suitable for investigating doped spin liquid physics.


\section{Objectives}
Measure the following:
\begin{itemize}
	\item Raman spectra of $\alpha$-ruthenium trichloride (hereafter \rucl) under ambient conditions as a function of doping by electrolyte gating (complete).
	\item Transport properties of \rucl as a function of doping by electrolyte gating (in progress).
	\item Interlayer separation as a function of doping by electrolyte gating using x-ray diffraction (in progress)
\end{itemize}
To conclude:
\begin{itemize}
	\item Up to an induced hole density of $10^{14}$ $\frac{1}{\text{cm}^2}$ without lattice disorder, \rucl remains an insulator (i.e., does not undergo an electronic phase transition).
\end{itemize}

\section{Motivation}
Quantum Spin Liquids (QSLs) are ground states of magnetic systems showing massive entanglement and correlation between spins without associated long-range magnetic order or spontaneous symmetry breaking, even at $T = 0$ K. 

QSLs have attracted research attention as these states host fractionalized excitations with nonabelian statistics \cite{Balents2010} which are relevant to quantum computation. QSLs may also be related to an unconventional form of superconductivity originally proposed by Anderson \cite{Lee2008}. $\alpha$-ruthenium trichloride (hereafter RuCl\textsubscript{3}) is a leading candidate material for having a ground state described as a Kitaev QSL, an exactly solvable quantum spin liquid state arising from anisotropic exchange interactions on a honeycomb lattice \cite{Kitaev2006}.

I propose investigating \rucl using electrolyte gating, transport, Raman spectroscopy, and x-ray diffraction to rule out the presence of an electronic phase transition below a certain charge density.

\section{Literature Review}
Measurements of \rucl are consistent with it being a 2D material \cite{Kim2015a} and a spin-assisted Mott insulator \cite{Plumb2014}, with either thermally-activated band conduction and electron charge carriers \cite{Rojas1983}, or variable range hopping conduction \cite{Mashhadi2018}. Measurements of the band gap in \rucl fall into two broad categories. Scanning tunneling spectroscopy and thermal activation of conductivity suggest a value for $E_{G}$ between of around 0.2 eV \cite{Binotto1971,Rojas1983,Ziatdinov2016}. Other measurements report larger gaps, including electron spectroscopy (between 1.2 eV and 1.9 eV \cite{Koitzsch2016,Zhou2016,Sinn2016}), and optical methods (between 0.2 eV and 1.2 eV \cite{Reschke2017,Pollini1996,Sandilands2016}). Calculations expect a gap between 0.05 eV and 1 eV  \cite{Sarikurt2017,Kim2015}.

Magnetically, \rucl is a paramagnet above its N{\'e}el temperature of 7 K; below 7 K it orders in an antiferromagnetic zigzag phase \cite{Sears2015}. Magnetic excitations above the N{\'e}el temperature, as observed by neutron scattering and raman scattering \cite{Banerjee2016,Sandilands2015}, are consistent with predictions for a Kitaev QSL, suggesting that \rucl is a proximate spin liquid.

\rucl displays several phase transitions. At approximately 150K, \rucl  undergoes a hysteretic phase transition from a rhombohedral to a monoclinic crystal structure \cite{Kubota2015,Ziatdinov2016,Glamazda2017,Reschke2017}. In addition to the N{\'e}el ordering at 7 K, another magnetic transition occurs at 14 K in some samples due to stacking faults \cite{Banerjee2016}. Under application of an in-plane magnetic field greater than $H_{c} \approx 7$ T, the antiferromagnetic phase is suppressed and the ground state becomes a field-induced disordered state. \cite{Hentrich2017,Wolter2017,Banerjee2017,Wang2017,Baek2017}. Measurements of the excitations of this disordered state by neutron scattering, microwave absorption, and thermal conductivity suggest this state is a field-induced quantum spin liquid \cite{Banerjee2016,Wellm2017,Kasahara2018}. Recently, a novel type of field-induced spin liquid state in \rucl was identified \cite{Lampen-Kelley2018}.

Raman studies of bulk and exfoliated \rucl show a spectrum that is thickness independent above ~40 nm, with several characteristic peaks below 500 cm$^{-1}$. For samples thinner than 10 nm, the three Raman peaks near 300 cm$^{-1}$ begin to merge. The presence of a broad background signal at low Raman shifts has been interpreted as phonon coupling to magnetic excitations of the proximate spin liquid state \cite{Zhou2018,Du2019}.

When exposed to ions in solution, cations easily intercalate between the van der Waals layers where two chlorine planes meet. Applying an electric potential to \rucl in an ion-containing solution can reversibly drive cations in and out of the spaces between layers. For intercalated \rucl, X-ray diffraction measurements show the increase in interlayer separation is consistent with the size of the intercalated ion \cite{Steffen1986,Schollhorn1983}.

There have been several previous attempts to dope \rucl and other QSL materials and candidates. Zhou did not observe a phase transition when doping \rucl by intercalation of charge donors or evaporating them onto the surface of bulk \rucl \cite{Zhou2016}. For similarly-intercalated \ruclnospace , Koitzsch did not observe a phase transition but instead posited that the intercalation caused charge ordering that blocked nearest-neighbor Kitaev interactions \cite{Koitzsch2017a}. However, the absence of a phase transition in these measurements may be due to insufficient surface doping (for Zhou's measurement) or lattice distortions caused by intercalation (for both Zhou and Koitzsch). Efforts to chemically dope Herbertsmithite \cite{Kelly2016} and the QSL candidate LiZn$_{2-x}$Mo$_{3}$O$_{8}$ \cite{Sheckelton2015} also did not show any phase transitions, though again lattice distortions cannot be ruled out. Notably, first principles density functional calculations suggest electrons added to a Kagome lattice-type quantum spin liquid will become localized due to lattice distortions \cite{Liu2018}.

\section{Previous Work in DGG Lab}

\begin{figure}
\centering
\begin{minipage}{0.3\textwidth}
\centering 
  {\includegraphics[width=0.95\textwidth]{C:/Users/dsbjr/Documents/DissertationProposal/images/thermalactivation}\label{fig:f2}}
  \captionsetup{width=0.9\textwidth}
  \captionof{figure}{Thermal activation energy extracted from an Arrhenius plot of conductivity vs temperature for exfoliated RuCl3 gated with DEME-TFSI}
\end{minipage}%
\begin{minipage}{0.3\textwidth}
\centering
  {\includegraphics[width=0.95\textwidth]{C:/Users/dsbjr/Documents/DissertationProposal/images/roomTempGateSweepsAnfractuousPopsicle}\label{fig:f1}}
  \captionsetup{width=0.9\textwidth}
  \captionof{figure}{Room temperature ionic liquid gate sweeps}
\end{minipage}%
\begin{minipage}{0.3\textwidth}
\centering
	{\includegraphics[width=0.95\textwidth]{C:/Users/dsbjr/Documents/DissertationProposal/images/Rookle09VRHdata.jpg}\label{fig:f3}}
	\captionsetup{width=0.9\textwidth}
  \captionof{figure}{Resistance-temperature curve for exfoliated \rucl sample showing variable range hopping}
\end{minipage}
\end{figure}

\subsection{Transport}

Samples with thicknesses from bulk to tens of nanometers measured in the DGG lab (Figure 1) were found to have resistivity and thermal activation energies approximately consistent with the literature \cite{Rojas1983}. Direct electrolyte gating with DEME-TFSI showed only minor changes in resistivity and thermal activation energy. When the surface of \rucl is protected by a thin layer of hexagonal boron nitride and an ionic liquid gate (DEME-TFSI) is applied (Figure 2), I continued to find simply activated conduction and small variations in room temperature resistance and thermal activation energy.

Subsequent measurements in the DGG lab show that in unbiased \rucl exposed to electrolyte for an extended period (i.e., hours), cations intercalate between the layers and cannot be removed even with a subsequent bias up to the electrochemical stability window of the electrolyte. Because the measurements discussed in the previous paragraph were made after pumping overnight to remove water, these samples were most likely fully intercalated independent of applied bias, and therefore not representative of an unperturbed lattice. More recent measurements, in which the bias is applied during pumpdown such that the material is not intercalated, suggest that the material is more resistive than pristine \ruclnospace. Quantitative data for these measurements is forthcoming.

A thin (3 nm) sample of \rucl displayed variable range hopping rather than simply activated conduction (Figure 3), with a thermal activation energy in the high temperature limit of 53 meV, consistent with previously published work for thin flakes of \rucl \cite{Mashhadi2018}. Applying an ionic liquid gate to this thin sample made only small changes to its properties. This is the only sample I have measured (one of ten) that shows variable range hopping.

\subsection{Raman Spectroscopy}

\rucl has seven Raman spectroscopy peaks; six of these are intralayer vibrational modes consistent with the symmetries of the \rucl lattice. The seventh does not have an associated symmetry but may be associated with an interlayer vibration \cite{Zhou2018}. Raman spectra of bulk and exfoliated ruthenium chloride measured by the DGG group are consistent with that reported in literature.

When exposed to electrolyte under ambient conditions, cations readily intercalate between van der Waals layers and dramatically change the Raman spectrum of the material. Peaks near 300 cm$^{-1}$ are broadened into a single diffuse peak, and the peak near 150 cm$^{-1}$ disappears. Applying a large negative gate voltage to the eletrolyte reverses this intercalation by driving the cations from the lattice and returns the raman spectrum to its previous, non-intercalated state. For \rucl under electrolyte biased at large negative voltage such that cations are not intercalated, the Raman peak positions and peak widths are identical to pristine \rucl within a Raman shift of less than 1 cm$^{-1}$. Because the Raman spectra of \rucl with electrolyte gating and negative bias is identical to pristine \ruclnospace , there is minimal distortion in single layers of \rucl despite the presence of the electrolyte.

\begin{figure}
\centering
\begin{minipage}{0.3\textwidth}
\centering 
  {\includegraphics[width=0.95\textwidth]{C:/Users/dsbjr/Documents/DissertationProposal/images/rucl3-RamanSpectraBulkGating.jpg}\label{fig:f4}}
  \captionsetup{width=0.9\textwidth}
  \captionof{figure}{Raman spectra under ambient conditions of pristine \rucl and \rucl intercalated by DEME-BF4, offset for clarity}
\end{minipage}%
\begin{minipage}{0.3\textwidth}
\centering
  {\includegraphics[width=0.95\textwidth]{C:/Users/dsbjr/Documents/DissertationProposal/images/RuCl3_Pristine_Fit.png}\label{fig:f5}}
  \captionsetup{width=0.9\textwidth}
  \captionof{figure}{Raman spectrum for pristine \rucl with Lorentzian peak fits}
\end{minipage}%
\begin{minipage}{0.3\textwidth}
\centering
	{\includegraphics[width=0.95\textwidth]{C:/Users/dsbjr/Documents/DissertationProposal/images/RuCl3_DEME-BF4_IntercalatedFit.png}\label{fig:f6}}
	\captionsetup{width=0.9\textwidth}
  \captionof{figure}{Raman spectra for \rucl gated with DEME-BF4 at large negative gate voltage. Note the similarity to the spectrum for pristine \ruclnospace}
\end{minipage}
\end{figure}


\subsection{X-ray diffraction}


\section{Research and Methodology}

My previous measurements suggest that charge carriers added to \rucl do not cause substantial changes in transport properties, even if the doping method does not alter the chemistry or lattice structure. Further, a review of the literature suggests that (1) charge induced in \rucl and other frustrated magnetic QSL candidates does not result in significant change in their transport properties and (2) induced charges may localize in these materials, preventing any electronic phase transitions. I think the charge I have added to \rucl localizes and therefore transport properties remain largely unchanged.

DFT calculations by Liu suggest that added charge is localized by forming polarons bound to particular lattice sites \cite{Liu2018}. Specifically, electrostatic interactions cause the lattice around an added charge carrier to relax, creating a local potential well that immobilizes the charge carrier. This lattice relaxation is predicted to cause changes in bond lengths, which should result in changes in the Raman spectrum (I would expect peak broadening and/or a shift in peak position). I propose to find evidence of localization by looking for a doping-dependent feature in the Raman spectrum of \rucl using ionic liquid gating.

\subsection{Doping Method}

Ionic liquid gating uses an electrically polarized molten salt to induce a high charge density at the surface of a material of interest. The potential at the surface is proportional to the ratio of the ion charge to the size of the anion or cation. Because the anion and cation size is $\mathcal{O}$(1 nm), the electric field at the interface is high enough to induce charge densities of $\mathcal{O}$(10$^{14}$ cm$^{-2}$). Ionic liquid gating has been used to demonstrate phase transitions and investigate the structure of other 2D materials\cite{Braga2012,Ueno2008}.

I will measure the charge density by integrating the current applied to the ionic liquid and measuring the area visually. Hall measurements on \ruclnospace, which is highly insulating, have been challenging.

\subsection{Raman Measurements}

I will measure the Raman spectra of exfoliated \rucl samples through an ionic liquid. Raman measurements will be made on the Horiba Labram in SNC under ambient (room temperature, 1 atm) conditions on devices fabricated for this particular purpose. I have confirmed that the Raman spectra of bulk and exfoliated \rucl are similar at thicknesses greater than 10 nm (Figure 4), as previously identified in the literature.

My most recent measurements show that certain Raman features in ionic liquid-gated \rucl become broadened or invisible when a positive voltage is applied, and return when a negative voltage is applied (Figure 5). The switching behavior is repeatable and robust. I have observed this effect in three separate samples measured using two different gating methods/geometries. This effect may be related to localization or may be some kind of phase transition.

\begin{figure}
\centering
\begin{minipage}{0.5\textwidth}
\centering
	{\includegraphics[width=0.95\textwidth]{C:/Users/dsbjr/Documents/DissertationProposal/images/exfoliatedvsbulkrucl3spectra.jpg}\label{fig:f4}}
  \captionsetup{width=0.9\textwidth}
  \captionof{figure}{Raman spectra for exfoliated and bulk \ruclnospace, offset for clarity}
\end{minipage}%
\begin{minipage}{0.5\textwidth}
\centering 
  {\includegraphics[width=0.95\textwidth]{C:/Users/dsbjr/Documents/GitHub/Dissertation.Proposal/images/BgPeaksDisappear.jpg}\label{fig:f5}}
  \captionsetup{width=0.9\textwidth}
  \captionof{figure}{Raman spectra measured with in-situ ionic liquid gating (DEME-BF4), offset for clarity. Red arrows are a guide to the eye for features that depend on gate voltage}
\end{minipage}
\end{figure}

\subsection{Transport Measurements}

Thermal activation energies extracted from resistance-temperature curves of \rucl at fixed ionic liquid voltages show only minor changes as a function of gate voltage. However, room temperature gate sweeps show some small repeatable structure, which may be related to charge localization.

I will make in-situ resistivity measurements of exfoliated \rucl while measuring Raman spectra. I will compare changes in resistivity to changes in Raman spectra as a function of gate voltage and draw conclusions about localization from the comparison.

\subsection{Experimental Controls}

I will use MoS$_{2}$ as a positive experimental control. Monolayer MoS$_{2}$ undergoes a metal-insulator transition at a carrier density of approximately 1 x $10^{13}$ cm$^{-2}$ \cite{Radisavljevic2013}, readily achievable using ionic liquid gating. Using the same fabrication methods I will use for other samples, I will show that ionic liquid gating induces the expected metal-insulator transition in MoS$_{2}$, verifying that the technique works.

\subsection{Outcomes}

This series of measurements has several possible outcomes, discussed below

\subsubsection{Observe a doping-dependent feature in Raman spectra}
I think this is the most likely outcome. In this case, I would forward the Raman spectra I've measured to a DFT collaborator (TBD) to determine if the Raman shift is consistent with localization. If I do not observe a doping-dependent feature then:

\subsubsection{Do not observe a doping-dependent feature in Raman spectra}
This is the least desirable outcome. If the changes I've observed are experimental artifacts, that could mean either (1) the density of localized states is much smaller than the charge density that should be induced by the ionic liquid, or (2) localization does not occur. I would rule out (1) by looking at the measured carrier density, perhaps by pursuing Hall measurements, even though they are challenging. Then I would conclude that neither localization nor any electronic transitions happen below the highest measured carrier density.

\section{Risks}

\subsection{Electrochemistry}
Electrochemistry between the ionic liquid, ohmic contacts, and \rucl could confuse the measured or calculated carrier density induced in the \ruclnospace. To minimize any electrochemical effects, I will dry the ionic liquid prior to use, use only materials known not to react with the ionic liquid, and carefully monitor the leakage current during measurement.

\subsection{Large Raman Background}
The ionic liquid may have features in its Raman spectra that could obscure localization features. Accordingly, I have identified an ionic liquid (DEME-BF4) with no Raman features in the region of interest. To further reduce the impact of an ionic liquid Raman background, I will use the thinnest layer of ionic liquid possible and minimize the noise by taking multiple spectra.

\subsubsection{Fabrication}

The notional fabrication process will be:

\begin{itemize}
	\item Exfoliate \rucl using scotch tape onto 300 nm SiO\textsubscript{2} on Si chips using O\textsubscript{2} plasma and heat
	\item Select and characterize flakes using optical and atomic force microscopy
	\item Write and develop bond pads and ohmic contact pattern via electron beam lithography
	\item Argon milling to reduce contact resistance
	\item Evaporate Ti/Au contacts
	\item Liftoff in Acetone
	\item Wirebond and apply ionic liquid
\end{itemize}

\subsubsection{Measurement}

The notional measurement process will be:

\begin{itemize}
	\item Measure Raman spectrum without ionic liquid applied
	\item Measure Raman spectrum through grounded ionic liquid
	\item Measure \rucl resistivity, gate current, and Raman spectra at various gate voltages in the electrochemical stability window of the ionic liquid. 
\end{itemize}

\section{Timeframe}
I plan to graduate in June 2019. A notional schedule follows.

\begin{itemize}
	\item 28 January to 4 March 2019: Prototype and refine measurement scheme
	\item 4 March to 29 March 2019: Final measurements for publication/dissertation quality data using refinements from previous month
	\item 30 March to 17 April 2019: Data analysis
	\item 18 April to 16 May 2019: Write dissertation and submit to committee
	\item Week of 27 May 2019: Dissertation defense
	\item 6 June 2019: Submit dissertation to Stanford
	\item 17 June 2019: Degree conferred	
\end{itemize}


\bibliography{C:/Users/dsbjr/Documents/GitHub/Dissertation.Proposal/RuCl3Bibliography}

\end{document}
