\documentclass[11pt]{article}
\usepackage{amsmath}
\usepackage{amsfonts}
\usepackage{amssymb}
\usepackage{gensymb}
\usepackage{cite}
\usepackage{graphicx}
\usepackage{multicol,caption}
\usepackage[margin=0.5in]{geometry}
\usepackage[stable]{footmisc}
\usepackage{comment}
\usepackage{flushend}

\bibliographystyle{unsrt}

\newenvironment{Figure}
  {\par\medskip\noindent\minipage{\linewidth}}
  {\endminipage\par\medskip}

\setlength{\columnsep}{0.25in}

\raggedbottom

\begin{document}
\newcommand{\rucl}{RuCl\textsubscript{3} }
\newcommand{\ruclnospace}{RuCl\textsubscript{3}}
\graphicspath{C:/Users/dsbjr/Documents/DissertationProposal/images}
\pagenumbering{gobble}

\Large
\begin{center}

\Huge Dissertation Proposal: Electrolyte Gating of the Two-Dimensional Proximate Kitaev Spin Liquid $\alpha$-Ruthenium Trichloride\\

\hspace{10pt}

% Author names and affiliations
\large
Derrick S. Boone, Jr.$^1$\\
	
\hspace{10pt}

\small  
$^1$ Ph.D. Candidate, Department of Applied Physics, Stanford University\\
dsbjr@stanford.edu\\

\end{center}

\section*{Executive Summary}
Quantum Spin Liquids (QSLs) are magnetic states of matter characterized by massive ground state entanglement without long-range magnetic order, even at $T = 0$ K. These materials host exotic quasiparticle excitations (e.g., Majorana fermions) and, upon doping, may demonstrate unconventional superconductivity. Recently, $\alpha$-ruthenium trichloride has attracted research interest because it is a Mott insulator whose ground state closely approximates a Kitaev QSL. Accordingly, doping $\alpha$-ruthenium trichloride offers an opportunity to explore both Mott insulator physics and predictions for QSLs. Previous attempts to dope $\alpha$-ruthenium trichloride have either disordered the lattice, added only a small number of charge carriers, or both. To avoid these issues, I have doped $\alpha$-ruthenium trichloride using electrolyte gating and performed in-situ transport, Raman spectroscopy, and x-ray diffraction measurements. I have found that cations in the electrolyte readily intercalate between the layers, despite their large size. Independent of intercalation, I do not observe a Mott metal-insulator transition (or any electronic transition) within the limits of electrolyte gating - in fact, the transport properties of $\alpha$-ruthenium trichloride are largely insensitive to doping. I propose to extend existing research by showing that $\alpha$-ruthenium trichloride does not undergo any electronic phase transition, even in the limit of a pristine lattice and large doping . This result suggests that the electronic structure of $\alpha$-ruthenium trichloride requires further investigation and that the material may not be suitable for investigating doped spin liquid physics.

\section{Objectives}
Measure the following:
\begin{itemize}
	\item Raman spectra of $\alpha$-ruthenium trichloride (hereafter \ruclnospace) under ambient conditions as a function of doping by electrolyte gating (complete).
	\item Transport properties of \rucl as a function of doping by electrolyte gating (nearly complete).
	\item Interlayer separation as a function of doping by electrolyte gating using x-ray diffraction (in progress)
\end{itemize}
To conclude one of the following:
\begin{enumerate}
	\item Observe an electronic phase transition in \rucl as a function of doping.
	\item Do not observe an electronic phase transition in \rucl as a function of doping and identify lattice distortions from Raman spectroscopy and/or x-ray diffraction (XRD) that may disturb Kitaev interactions.
	\item Do not observe an electronic phase transition or lattice distortions as a function of doping, and rule out the presence of either below a particular charge density.
\end{enumerate}

\section{Motivation}
Quantum Spin Liquids (QSLs) are ground states of magnetic systems showing massive entanglement and correlation between spins without associated long-range magnetic order or spontaneous symmetry breaking, even at $T = 0$ K. 

QSLs have attracted research attention because these states host fractionalized excitations with nonabelian statistics \cite{Balents2010} which are relevant to quantum computation. QSLs may also be related to an unconventional form of superconductivity originally proposed by Anderson \cite{Lee2008}. \ruclnospace , a Mott insulator, is a leading candidate material for having a ground state approximating a Kitaev QSL, an exactly solvable quantum spin liquid state arising from anisotropic exchange interactions on a honeycomb lattice \cite{Kitaev2006}.

Adding charge to Mott insulators has resulted in unexpected physics, most notably high-temperature superconductors. Further, the connection between \rucl and fractionalized excitations makes doped \rucl an exciting material to study. I therefore propose investigating \rucl using electrolyte gating, transport, Raman spectroscopy, and XRD. I have selected these tools to add charge to the material while simultaneously characterizing the electronic phase (transport) and changes to the lattice (Raman spectroscopy and XRD).

\section{Literature Review}
Measurements of \rucl are consistent with it being a 2D material \cite{Kim2015a} and a spin-assisted Mott insulator \cite{Plumb2014}, with either thermally-activated band conduction \cite{Rojas1983}, or variable range hopping conduction \cite{Mashhadi2018}. Measurements of the gap in \rucl are inconsistent and depend on the method used. Scanning tunneling spectroscopy and thermal activation of conductivity suggest a value for $E_{G}$ around 0.2 eV \cite{Binotto1971,Rojas1983,Ziatdinov2016}. Other measurements report larger gaps, including electron spectroscopy (between 1.2 eV and 1.9 eV \cite{Koitzsch2016,Zhou2016,Sinn2016}), and optical methods (between 0.2 eV and 1.2 eV \cite{Reschke2017,Pollini1996,Sandilands2016}). Calculations suggest a gap between 0.05 eV and 1 eV  \cite{Sarikurt2017,Kim2015}.

Magnetically, \rucl is a paramagnet above its N{\'e}el temperature of 7 K; below 7 K it orders in an antiferromagnetic zigzag phase \cite{Sears2015}. Magnetic excitations of \rucl above and below the N{\'e}el temperaure, measured by neutron scattering and raman scattering \cite{Banerjee2016,Sandilands2015}, are not described by linear spin wave theory, a semiclassical theory that is only accurate in the limit of weak quantum fluctuations.  Instead, a quantum model that includes Kitaev interactions is required to describe the magnetic excitations, confirming \rucl as a proximate Kitaev QSL.

\rucl displays several phase transitions. At approximately 150K, \rucl  undergoes a hysteretic phase transition from a rhombohedral to a monoclinic crystal structure \cite{Kubota2015,Ziatdinov2016,Glamazda2017,Reschke2017}. In addition to the N{\'e}el ordering at 7 K, another magnetic transition occurs at 14 K in some samples due to stacking faults \cite{Banerjee2016}. Under application of an in-plane magnetic field greater than $H_{c} \approx 7$ T, the antiferromagnetic phase is suppressed and the ground state becomes a field-induced disordered state \cite{Hentrich2017,Wolter2017,Banerjee2017,Wang2017,Baek2017}. Measurements of the excitations of this disordered state by neutron scattering, microwave absorption, and thermal conductivity suggest this state is a field-induced quantum spin liquid \cite{Banerjee2016,Wellm2017,Kasahara2018}. Recently, a novel type of field-induced spin liquid state in \rucl was identified \cite{Lampen-Kelley2018}.

Raman studies of bulk and exfoliated \rucl show a spectrum that is thickness independent for thicknesses greater than approximately 40 nm, with several characteristic peaks below 500 cm$^{-1}$. For samples thinner than 10 nm, three Raman peaks near 300 cm$^{-1}$ begin to merge. The presence of a broad background signal at low Raman shifts has been interpreted as phonon coupling to magnetic excitations of the proximate spin liquid state \cite{Zhou2018,Du2019}.

When exposed to solutions containing ions, cations easily intercalate between electronegative chlorine planes of \rucl van der Waals layers. Applying an electric potential to \rucl in an ion-containing solution can reversibly drive cations in and out of the spaces between layers. For intercalated \ruclnospace , XRD measurements show the increase in interlayer separation is consistent with the size of the intercalated ion \cite{Steffen1986,Schollhorn1983}.

There have been several previous attempts to dope \rucl and other QSL materials and candidates. Zhou, \textit{et al.} did not observe a phase transition when doping \rucl by intercalation of charge donors or evaporating them onto the surface of bulk \rucl \cite{Zhou2016}. For similarly-intercalated \ruclnospace , Koitzsch \textit{et al.} did not observe a phase transition and also suggested that the intercalation caused charge ordering that blocked nearest-neighbor Kitaev interactions \cite{Koitzsch2017a}. However, the absence of a phase transition in these measurements may be due to insufficient surface doping (for Zhou's measurement) or lattice distortions caused by intercalation (for both Zhou and Koitzsch). Efforts to chemically dope QSL materials Herbertsmithite \cite{Kelly2016} and LiZn$_{2-x}$Mo$_{3}$O$_{8}$ \cite{Sheckelton2015} also did not show any phase transitions, though again lattice distortions cannot be ruled out. Notably, first principles density functional calculations suggest electrons added to a Kagome lattice-type quantum spin liquid will become localized due to lattice distortions \cite{Liu2018}. This kind of localization, if it exists, may explain the absence of a phase transition.


\section{Research and Methodology}

A review of the literature suggests that \rucl does not undergo an electronic phase transition upon doping. However, previous attempts to dope \rucl have not approached the charge density possible with electrolyte gating, or have involved a technique that necessarily distorts the lattice. These lattice distortions may affect the Kitaev interactions that make \rucl a QSL, and therefore prevent investigation of properties related to QSLs.

Electrolyte gating provides an opportunity to heavily dope \rucl without disordering the lattice. Therefore, measuring transport with electrolyte gating allows characterizing the electronic phase as a function of doping independent of the disorder inherent in other methods. Raman spectroscopy and XRD can be used to interrogate the lattice and confirm or exclude lattice distortion.

I propose drawing one of the following conclusions from the results of the measurements recommended in section 1:
\begin{enumerate}
	\item Observe an electronic phase transition in \rucl as a function of doping.
	\item Do not observe an electronic phase transition in \rucl as a function of doping and identify lattice distortions from Raman spectroscopy or XRD that could disturb Kitaev interactions.
	\item Do not observe an electronic phase transition or lattice distortions as a function of doping, and rule out the presence of either below a particular charge density.
\end{enumerate}
	

\subsection{Doping Method}

Electrolyte gating uses an electrically polarized molten salt to induce a high charge density at the surface of a material of interest. The potential at the surface is proportional to the ratio of the ion charge to the size of the anion or cation. Because the anion and cation size is $\mathcal{O}$(1 nm), the electric field at the interface is high enough to induce charge densities of $\mathcal{O}$(10$^{14}$ cm$^{-2}$). Ionic liquid gating has been used to demonstrate phase transitions and investigate the structure of other 2D materials\cite{Braga2012,Ueno2008}.

I will measure the induced charge density by either (1) integrating the current applied to the ionic liquid and measuring the area visually or (2) electrochemical impedance spectroscopy. Hall measurements on \ruclnospace, which is highly insulating, have been challenging.

\subsection{Raman Measurements}

I have measured the Raman spectra of exfoliated \rucl samples through an electrolyte. Raman measurements were made on the Horiba Labram in SNC under ambient (room temperature, 1 atm) conditions on devices fabricated for this particular purpose. Raman peaks shifting, broadening, or disappearing could indicate lattice distortions or phase transitions. I have confirmed that the Raman spectra of bulk and exfoliated \rucl are similar at thicknesses greater than 10 nm (Figure 4), as previously identified in the literature.

\begin{figure}
\centering
	{\includegraphics[width=0.5\textwidth]{C:/Users/dsbjr/Documents/DissertationProposal/images/exfoliatedvsbulkrucl3spectra.jpg}}
  \captionsetup{width=0.5\textwidth}
  \captionof{figure}{Raman spectra for exfoliated and bulk \ruclnospace, offset for clarity}
  \label{fig:f1}
\end{figure}

\subsection{Transport Measurements}

I have measured resistivity and thermal activation energies extracted from resistance-temperature curves of \rucl at fixed ionic liquid voltages in a DGG group cryostat. Phase changes could include a metal-insulator transition (identified by a change in sign of $\frac{d\rho}{dT}$), or an abrupt change in the functional form or the activation energy of $\rho(T)$.

\subsection{X-ray Measurements}

I will measure the diffraction pattern of pristine and electrolyte gated \rucl as a function of electrolyte gate voltage in the Stanford XRD lab. Changes to the interlayer separation or broadening of the associated peak could indicate lattice distortion or a phase transition.

\section{Previous Work in DGG Lab}

\begin{figure}
\centering
\centering 
  {\includegraphics[width=0.5\textwidth]{C:/Users/dsbjr/Documents/DissertationProposal/images/thermalactivation}}
  \captionsetup{width=0.5\textwidth}
  \captionof{figure}{Thermal activation energy extracted from an Arrhenius plot of conductivity vs temperature for exfoliated RuCl3 gated with DEME-TFSI}
  \label{fig:f2}
\end{figure}
\begin{figure}
\centering
  {\includegraphics[width=0.5\textwidth]{C:/Users/dsbjr/Documents/DissertationProposal/images/roomTempGateSweepsAnfractuousPopsicle}}
  \captionsetup{width=0.5\textwidth}
  \captionof{figure}{Room temperature ionic liquid gate sweeps}
  \label{fig:f3}
\end{figure}
\begin{figure}
\centering
	{\includegraphics[width=0.5\textwidth]{C:/Users/dsbjr/Documents/DissertationProposal/images/Rookle09VRHdata.jpg}}
	\captionsetup{width=0.5\textwidth}
  \captionof{figure}{Resistance-temperature curve for exfoliated \rucl sample showing variable range hopping}
  \label{fig:f4}
\end{figure}

\subsection{Transport}

Samples with thicknesses from bulk to tens of nanometers I measured in the DGG lab (Figure ~\ref{fig:f2}) were found to have resistivity and thermal activation energies approximately consistent with the literature \cite{Rojas1983}. Direct electrolyte gating with DEME-TFSI showed only minor changes in resistivity and thermal activation energy. When the surface of \rucl is protected by a thin layer of hexagonal boron nitride and an ionic liquid gate (DEME-TFSI) is applied (Figure ~\ref{fig:f3}), I continued to find simply activated conduction and small variations in room temperature resistance and thermal activation energy.

Subsequent measurements in the DGG lab show that in unbiased \rucl exposed to electrolyte for an extended period (i.e., hours), cations intercalate between the layers and cannot be removed even with a subsequent bias up to the electrochemical stability window of the electrolyte. Because the measurements discussed in the previous paragraph were made after pumping overnight to remove water, these samples were most likely fully intercalated independent of applied bias, and therefore not representative of an unperturbed lattice. I have since repeated this measurement with bias applied during pumpdown and confirmed that the material was not intercalated. Results from this measurement continue to show only minor changes in thermal activation energy and resistivity, confirming the absence of an electronic phase transition even with an unperturbed lattice.

A thin (3 nm) sample of \rucl displayed variable range hopping rather than simply activated conduction (Figure ~\ref{fig:f4}), with a thermal activation energy in the high temperature limit of 53 meV, consistent with previously published work for thin flakes of \rucl \cite{Mashhadi2018}. Applying an ionic liquid gate to this thin sample made only small changes to its properties. This is the only sample I have measured (one of ten) that shows variable range hopping.

\subsection{Raman Spectroscopy}

\rucl has seven Raman peaks; six of these are intralayer vibrational modes consistent with the symmetries of the \rucl lattice. The seventh does not have an associated symmetry but may be associated with an interlayer vibration \cite{Zhou2018}. Raman spectra of bulk and exfoliated \rucl I measured are consistent with that reported in literature.

When exposed to electrolyte under ambient conditions, cations readily intercalate between van der Waals layers and dramatically change the Raman spectrum of the material. Peaks near 300 cm$^{-1}$ are broadened into a single diffuse peak, and the peak near 150 cm$^{-1}$ disappears. Applying a large negative gate voltage to the eletrolyte reverses this intercalation by driving the cations from the lattice and returns the Raman spectrum to its previous, non-intercalated state. For \rucl under electrolyte biased at large negative voltage such that cations are not intercalated, the Raman peak positions and peak widths are identical to pristine \rucl within a Raman shift of less than 1 cm$^{-1}$. Because the Raman spectra of \rucl with electrolyte gating and negative bias is identical to pristine \ruclnospace , there is minimal distortion in single layers of \rucl despite the presence of the electrolyte.

\begin{figure}
\centering 
  {\includegraphics[width=0.5\textwidth]{C:/Users/dsbjr/Documents/DissertationProposal/images/rucl3-RamanSpectraBulkGating.jpg}}
  \captionsetup{width=0.5\textwidth}
  \captionof{figure}{Raman spectra under ambient conditions of pristine \rucl and \rucl intercalated by DEME-BF4, offset for clarity}
  \label{fig:f5}
\end{figure}
\begin{figure}
\centering
  {\includegraphics[width=0.5\textwidth]{C:/Users/dsbjr/Documents/DissertationProposal/images/RuCl3_Pristine_Fit.png}}
  \captionsetup{width=0.5\textwidth}
  \captionof{figure}{Raman spectrum for pristine \rucl with Lorentzian peak fits}
  \label{fig:f6}
\end{figure}
\begin{figure}
\centering
{\includegraphics[width=0.5\textwidth]{C:/Users/dsbjr/Documents/DissertationProposal/images/RuCl3_DEME-BF4_IntercalatedFit.png}}
\captionsetup{width=0.5\textwidth}
  \captionof{figure}{Raman spectra for \rucl gated with DEME-BF4 at large negative gate voltage. Note the similarity to the spectrum for pristine \ruclnospace}
  \label{fig:f7}
\end{figure}

\subsection{X-ray diffraction}

X-ray diffraction (XRD) measurements show \rucl has an interlayer separation of about 5.6 \r{A} \cite{Schollhorn1983}. This interlayer separation shows up clearly in XRD measurements as a sharp peak near $2\theta = $ 15\degree, which I have observed here at Stanford.

\begin{figure}
 \centering
	{\includegraphics[width=0.5\textwidth]{C:/Users/dsbjr/Documents/DissertationProposal/images/PristineRuCl3XRD.jpg}}
  \captionsetup{width=0.5\textwidth}
  \captionof{figure}{X-ray diffractogram for pristine exfoliated \rucl measured at Stanford}
  \label{fig:f8}
\end{figure}

\section{Remaining Measurements}

\subsection{Transport Measurements}

A collaborator has provided a Polymer Ionic Liquid (PIL), a new type of electrolyte with a sterically hindered cation that should not intercalate. I will also measure transport using a PIL to investigate electron doping without lattice disorder. While this will not change my existing results, it may help expand them.

\subsection{X-ray Measurements}

I will measure X-ray diffraction patterns for electrolyte-gated \rucl as a function of gate voltage. I will confirm that the cations intercalate and that at large negative gate voltage the interlayer spacing returns to its pristine value.

\section{Risks}

\subsection{Electrochemistry}

Electrochemistry between the electrolyte, ohmic contacts, and \rucl could confuse the measured or calculated carrier density induced in the \ruclnospace. To minimize any electrochemical effects, I will dry the electrolyte prior to use, use only materials known not to react with the electrolyte, and carefully monitor the leakage current during measurement.

\subsection{Large Raman Background}

The electrolyte may have features in its Raman spectra that could obscure localization features. Accordingly, I have identified an ionic liquid (DEME-BF4) with no Raman features in the region of interest. To further reduce the impact of an ionic liquid Raman background, I will use the thinnest layer of ionic liquid possible and minimize the noise by taking multiple spectra.

\subsection{X-ray measurements}

X-ray measurements may show that the interlayer spacing does not return to its original value after the ions are removed from the interlayer spaces (the peak may correspond to a different interlayer separation or be broadened - indicating a range of interlayer separations). I think this is unlikely given previous measurements on this material. However, all Kitaev interactions are present in a single layer of \ruclnospace , and the disorder in a single layer should be captured by Raman spectroscopy, which does not show any peak shifts or broadening. Therefore, I will include the change or disorder in interlayer spacing as a potential cause for the absence of a transition, which is a novel conclusion. 

\subsubsection{Fabrication}

The notional fabrication process will be:

\begin{itemize}
	\item Exfoliate \rucl using scotch tape onto 300 nm SiO\textsubscript{2} on Si chips using O\textsubscript{2} plasma and heat
	\item Select and characterize flakes using optical and atomic force microscopy
	\item Write and develop bond pads and ohmic contact pattern via electron beam or optical lithography
	\item Argon milling to reduce contact resistance
	\item Evaporate Ti/Au contacts
	\item Liftoff in Acetone
	\item Wirebond and apply ionic liquid
\end{itemize}

\section{Timeframe}
I plan to graduate in September 2019. A notional schedule follows.

\begin{itemize}
	\item May 2019: Finish remaining measurements
	\item June - July 2019: Writing and defense preparation
	\item Week of 22 July 2019: Dissertation defense
	\item 26 September 2019: Degree conferred	
\end{itemize}


\bibliography{C:/Users/dsbjr/Documents/GitHub/Dissertation.Proposal/RuCl3Bibliography}

\end{document}
